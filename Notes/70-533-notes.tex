\documentclass[12pt]{article}

\title{Notes for the 70-533 certification}
\author{Martin Nyborg Thomsen}
\date{\today}
    
\begin{document}
\maketitle

\section{Chapter 1: Design and implement Azure App Service Web Apps}

\subsection{Skill 1.1: Deploy web apps}
TBW

\subsection{Skill 1.2: Configure web apps}
TBW

\subsection{Skill 1.3: Configure diagnostics, monitoring, and analytics}
TBW

\subsection{Skill 1.4: Configure web apps for scale and resilience}
TBW

\section{Chapter 2: Create and manage Compute Resources}

\subsection{Skill 2.1: Deploy workloads on Azure Resource Manager (ARM) virtual machines (VMs)}
TBW

\subsection{Skill 2.2: Perform con guration management}
TBW

\subsection{Skill 2.3: Design and implement VM Storage}

\subsubsection{Blob types}
All persistent disks for an Azure Virtual Machine are stored in blob storage of an Azure Storage account. There are three types of blob  les:
\begin{itemize}
    \item \textbf{Block blobs} are used to hold ordinary  les up to about 4.7 TB.
    \item \textbf{Page blobs} are used to hold random access  les up to 8 TB in size. These are used for the
    VHD  les that back VMs.
    \item \textbf{Append blobs} are made up of blocks like the block blobs, but are optimized for append operations. These are used for things like logging information to the same blob from multiple VMs.
\end{itemize}

\subsubsection{Storage accounts}
An Azure Storage account can be one of three types:
\begin{itemize}
    \item \textbf{Standard}: Can store all types of data. Uses magnetic media (HDD)
    \item \textbf{Premium}: Can store Page blobs. Uses SSDs. 
    \item \textbf{Blob}: Can store Block blobs and Append blobs. Access tier can be set to \textbf{Hot} or \textbf{Cold}: 
    \begin{itemize}
        \item \textbf{Hot} access tier is when storage costs more, but access costs less.
        \item \textbf{Cold} access tier is when storage costs less, but access costs more.
    \end{itemize}
\end{itemize}

\subsubsection{Storage account replication}
\begin{itemize}
    \item \textbf{Locally redundant storage (LRS)}: Each blob has three copies in the data center.
    \item \textbf{Geo-redundant storage (GRS)} Each blob has three copies in the data center, and is asynchronously replicated to a second region for a total of six copies. In the event of a failure at the primary region, Azure Storage fails over to the secondary region.
    \item \textbf{Read-access geo-redundant storage (RA-GRS)} The same as (GRS), except you can access the replicated data (read only) regardless of whether a failover has occurred.
    \item \textbf{Zone redundant storage (ZRS)} Each blob has three copies in the data center, and is asynchronously replicated to a second data center in the same region for a total of six copies. Note that ZRS is only available for block blobs (no VM disks) in general-purpose storage accounts. Also, once you have created your storage account and selected ZRS, you cannot convert it to use to any other type of replication, or vice versa.
\end{itemize}

\subsubsection{Azure disk types}
Azure VMs use three types of disks:
\begin{itemize}
    \item \textbf{Operating System Disk (OS Disk)}: The C drive in Windows or /dev/sda on Linux. This disk is registered as an SATA drive and has a maximum capacity of 2048 gigabytes (GB). This disk is persistent and is stored in Azure storage.
    \item \textbf{Temporary Disk}: The D drive in Windows or /dev/sdb on Linux. This disk is used for short term storage for applications or the system. Data on this drive can be lost in dur- ing a maintenance event, or if the VM is moved to a different host because the data is stored on the local disk.
    \item \textbf{Data Disk Registered as a SCSI drive}: These disks can be attached to a virtual machine, the number of which depends on the VM instance size. Data disks have a maximum capacity of 4095 gigabytes (GB). These disks are persistent and stored in Azure Storage.
\end{itemize}

\subsubsection{Azure disks - Unmanaged and Managed}
There are two types of disks in Azure: Managed or Unmanaged.
\begin{itemize}
    \item \textbf{Unmanaged disks}: With unmanaged disks you are responsible for ensuring for the correct distribution of your VM disks in storage accounts for capacity planning as well as availability. An unmanaged disk is also not a separate manageable entity. This means that you cannot take advantage of features like role based access control (RBAC) or resource locks at the disk level.
    \item \textbf{Managed disks}: Managed disks handle storage for you by automatically distributing your disks in storage accounts for capacity and by integrating with Azure Availability Sets to provide isolation for your storage just like availability sets do for virtual machines. Managed disks also makes it easy to change between Standard and Premium storage (HDD to SSD) without the need to write conversion scripts.
\end{itemize}

\subsection{Operating system images}
TBW

\subsection{Creating a VM from an image}
TBW

\subsection{Virtual machine disk caching}
TBW

\subsection{Planning for storage capacity}
TBW

\subsection{Implementing disk redundancy for performance}
TBW

\subsection{Disk encryption}
TBW

\subsection{Using the Azure File Service}
TBW

\end{document}